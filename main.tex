% for notes environment
\usepackage{xsavebox}
\usepackage{hyperref}
\usepackage{graphicx}
\usepackage{luatexja}
\usepackage[hiragino-pro,deluxe,nfssonly,jis2004]{luatexja-preset}
\usepackage{fontspec}
\usepackage{epigraph}
\usepackage{etoolbox}
\usepackage{tikz}
\usepackage{framed}
\usepackage{mathtools}
\usepackage{listings}
\usepackage{libertine}
\usepackage[libertine]{newtxmath}
\usepackage{bxcoloremoji}
\usepackage{xcolor}
\usepackage{diagbox}
\usepackage{caption}
\usepackage{appendixnumberbeamer}
\usepackage{multirow}
\usepackage{xpatch}
\usepackage{multicol}
\usepackage{tabularx}

\usetikzlibrary{fit}

\setmonofont{CMU Typewriter Text}

\definecolor{links}{HTML}{2A1B81}
\hypersetup{colorlinks,linkcolor=,urlcolor=links}

\usetheme{Boadilla}
\usecolortheme{seahorse}
% \usefonttheme{serif}


\xpatchcmd{\itemize}
  {\def\makelabel}
  {\ifnum\@itemdepth=1\relax
     \setlength\itemsep{1.2ex}% separation for first level
   \else
     \ifnum\@itemdepth=2\relax
       \setlength\itemsep{0.8ex}% separation for second level
       \setlength\topsep{1.2ex}
     \else
       \ifnum\@itemdepth=3\relax
         \setlength\itemsep{0.05ex}% separation for third level
         \setlength\topsep{0.8ex}
   \fi\fi\fi\def\makelabel
  }
 {}
 {}

\setbeamercolor{page number in head/foot}{bg=blue!10}
\setbeamertemplate{footline}{%
  \leavevmode%
  \hbox{%
    \begin{beamercolorbox}[wd=.3\paperwidth,ht=2.25ex,dp=1ex,center]{author in head/foot}%
      \usebeamerfont{author in head/foot}\insertshortauthor\hspace*{1ex}(\insertshortinstitute)
    \end{beamercolorbox}%
    \begin{beamercolorbox}[wd=.3\paperwidth,ht=2.25ex,dp=1ex,center]{title in head/foot}%
      \usebeamerfont{title in head/foot}\insertshorttitle
    \end{beamercolorbox}%
    \begin{beamercolorbox}[wd=.3\paperwidth,ht=2.25ex,dp=1ex,center]{date in head/foot}%
      \insertshortdate{} @ \InsertConferenceShort
    \end{beamercolorbox}%
    \begin{beamercolorbox}[wd=.1\paperwidth,ht=2.25ex,dp=1ex,center]{page number in head/foot}%
      \insertframenumber{} / \inserttotalframenumber\hspace*{1ex}
    \end{beamercolorbox}}%
  \vskip0pt%
}

\beamertemplatenavigationsymbolsempty

\setbeamertemplate{bibliography item}{\insertbiblabel}
\setbeamersize{description width=1cm}
\setbeamertemplate{items}[circle]
\setbeamertemplate{section in toc}[circle]
\setbeamertemplate{subsection in toc}{%
  \leavevmode\leftskip=2em
  {%
    \usebeamerfont*{itemize item}%
    \usebeamercolor{subsection number projected}%
    \color{bg}%
    \raise1.25pt\hbox{\donotcoloroutermaths$\bullet$}}%
  \hskip1.5ex\inserttocsubsection\par}

% Definitions for the title page
\newcommand*{\GitHub}[1]{%
  \gdef\InsertGitHub{#1}%
}
\newcommand*{\Email}[1]{%
  \gdef\InsertEmail{\href{mailto:#1}{#1}}%
}
\newcommand{\ConferenceImpl}[2][]{%
  \gdef\InsertConferenceShort{#1}%
  \gdef\InsertConference{#2}%
}
\makeatletter
\newcommand\Conference{\@dblarg\ConferenceImpl}
\makeatother
\setbeamerfont{title}{size=\huge, series=\bfseries, family=\mcfamily\rmfamily}
\setbeamercolor{title}{bg=white}
\setbeamerfont{subtitle}{size=\large, series=\mdseries, family=\gtfamily\sffamily}
\setbeamerfont{email}{size=\scriptsize, family=\ttfamily}
\setbeamercolor{email}{bg=white}
\setbeamerfont{date}{shape=\itshape, family=\rmfamily}
\setbeamerfont{vc}{size=\scriptsize, family=\ttfamily}
\setbeamercolor{vc}{bg=white}

\renewcommand{\figurename}{Fig}

\input{vc.tex}

\setbeamertemplate{title page}
{%
  \vbox{}
  \vfill
  \begingroup
    \centering
    \hrulefill\par%
    \vskip1ex\par%
    \begin{beamercolorbox}[sep=0pt,center,shadow=false,rounded=true]{title}
      \vfill
      \usebeamerfont{title}\inserttitle\par%
      \ifx\insertsubtitle\@empty%
      \else%
        \vskip0.5ex%
        {\usebeamerfont{subtitle}\usebeamercolor[fg]{subtitle}\insertsubtitle\par}%
      \fi%
      \vfill  
    \end{beamercolorbox}%
    \hrulefill\par%
    \vskip2ex%
    \begin{beamercolorbox}[sep=0pt,center,shadow=false,rounded=true]{author}
      \usebeamerfont{author}\insertauthor
    \end{beamercolorbox}
    \begin{beamercolorbox}[sep=0pt,center,shadow=false,rounded=true]{email}
      \usebeamerfont{email}\InsertEmail
    \end{beamercolorbox}
    \vskip0.1ex
    \begin{beamercolorbox}[sep=5pt,center,shadow=false,rounded=true]{institute}
      \usebeamerfont{institute}\insertinstitute
    \end{beamercolorbox}
    \begin{beamercolorbox}[sep=5pt,center,shadow=false,rounded=true]{date}
      \usebeamerfont{date}\insertdate \normalfont @ \InsertConference
    \end{beamercolorbox}
    \begin{beamercolorbox}[sep=0pt,center,shadow=false,rounded=true]{vc}
      \usebeamerfont{vc}
      \url{https://github.com/\InsertGitHub} (\texttt{\GITAbrHash})
    \end{beamercolorbox}
    % {\centering
    %   \href{https://creativecommons.org/licenses/by-nc/4.0/}{%
    %     \includegraphics[width=0.1\textwidth]{img/by-nc.pdf}%
    %   }%
    % }
    {\usebeamercolor[fg]{titlegraphic}\inserttitlegraphic\par}
  \endgroup
  \vfill
}
\setbeamertemplate{blocks}[rounded][shadow=false]

% ============ ここを消すとNote消える ================
\mode<handout>{%
  \usepackage{pgfpages}
  \setbeameroption{show notes on second screen=right}
  \setbeamertemplate{note page}{%
    \vspace{2ex}\insertnote%
  }
}
% ============ ここを消すとNote消える ================


\renewcommand{\kanjifamilydefault}{\gtdefault}

\setbeamertemplate{caption}[numbered]
\resetcounteronoverlays{lstlisting}
\definecolor{bluegray}{rgb}{0.4, 0.6, 0.8}
\DeclareCaptionFormat{listing}{{\color{bluegray}\lstlistingname}#2#3}
\captionsetup[lstlisting]{format=listing, font={footnotesize}}
\captionsetup[figure]{name={Figure}}
\captionsetup[table]{name={Table}}
\setbeamerfont{footnote}{size=\scriptsize}

\setmonofont[Ligatures=TeX]{CMU Typewriter Text}

\setbeamertemplate{items}[circle]

\newfontfamily\quotefont[Ligatures=TeX]{Linux Libertine O} % selects Libertine as the quote font

\newcommand*\quotesize{60} % if quote size changes, need a way to make shifts relative
% Make commands for the quotes
\newcommand*{\openquote}
   {\tikz[remember picture,overlay,xshift=0em,yshift=-3ex]
   \node (OQ) {\quotefont\fontsize{\quotesize}{\quotesize}\selectfont``};\kern0pt}

\newcommand*{\closequote}[1]
  {\tikz[remember picture,overlay,xshift=1ex,yshift={#1}]
   \node (CQ) {\quotefont\fontsize{\quotesize}{\quotesize}\selectfont''};}

\newcommand*\shadedauthorformat{\emph} % define format for the author argument

% Now a command to allow left, right and centre alignment of the author
\newcommand*\authoralign[1]{%
  \if#1l
    \def\authorfill{}\def\quotefill{\hfill}
  \else
    \if#1r
      \def\authorfill{\hfill}\def\quotefill{}
    \else
      \if#1c
        \gdef\authorfill{\hfill}\def\quotefill{\hfill}
      \else\typeout{Invalid option}
      \fi
    \fi
  \fi}
% wrap everything in its own environment which takes one argument (author) and one optional argument
% specifying the alignment [l, r or c]
%
\newenvironment{shadequote}[2][l]%
{\hspace{0.5ex}
\authoralign{#1}
\ifblank{#2}
   {\def\shadequoteauthor{}\def\yshift{-1ex}\def\quotefill{\hfill}}
   {\def\shadequoteauthor{\par\authorfill\shadedauthorformat{#2}}\def\yshift{3ex}}
\begin{quote}\normalfont\openquote}
{\shadequoteauthor\quotefill\closequote{\yshift}\end{quote}}

\input{./lib/footnotemark.tex}
\newcommand\ballcircle[1]{%
  {%
    \usebeamercolor{enumerate item}%
    \tikzset{beameritem/.style={circle,inner sep=0,minimum size=2ex,text=enumerate item.bg,fill=enumerate item.fg}}%
    \tikz[baseline=(n.base)]\node(n)[beameritem]{\sffamily#1};%
  }%
}
\newcommand\ballref[1]{%
  \ballcircle{\ref{#1}}%
}

\input{./lib/callout.tex}
\input{./lib/listings.tex}
\newenvironment{notes}
  {%
    \begin{xlrbox}{NotesBox}
    \begin{minipage}{.95\textwidth}
    \small\rmfamily\mcfamily
    \begin{itemize}
    \setlength{\itemindent}{0em}
    \setlength{\footnotesep}{5mm}
  }{%
    \end{itemize}
    \end{minipage}
    \end{xlrbox}
    \note{\theNotesBox}}

\def\AtSOne#1\csod{%
	\begin{array}{c|}
		\hline
		#1\\
		\hline
	\end{array}
}%
\def\AtSTwo#1,#2\csod{%
	\begin{array}{c|c|}
		\hline
		#1 & #2\\
		\hline
	\end{array}
}%
\def\AtSThree#1,#2,#3\csod{%
	\begin{array}{c|c|c|}
		\hline
		#1 & #2 & #3\\
		\hline
	\end{array}
}%
\def\AtSFour#1,#2,#3,#4\csod{%
	\begin{array}{c|c|c|c|}
		\hline
		#1 & #2 & #3 & #4\\
		\hline
	\end{array}
}%
\def\AtSFive#1,#2,#3,#4,#5\csod{%
	\begin{array}{c|c|c|c|c|}
		\hline
		#1 & #2 & #3 & #4 & #5\\
		\hline
	\end{array}
}%
\def\AtSSix#1,#2,#3,#4,#5,#6\csod{%
	\begin{array}{c|c|c|c|c|c|}
		\hline
		#1 & #2 & #3 & #4 & #5 & #6\\
		\hline
	\end{array}
}
\newcommand{\SOne}[1]{\AtSOne#1\csod}
\newcommand{\STwo}[1]{\AtSTwo#1\csod}
\newcommand{\SThree}[1]{\AtSThree#1\csod}
\newcommand{\SFour}[1]{\AtSFour#1\csod}
\newcommand{\SFive}[1]{\AtSFive#1\csod}
\newcommand{\SSix}[1]{\AtSSix#1\csod}
\newcommand\card[2]{%
  \setlength{\fboxsep}{0pt}%
  \fcolorbox{black}{#1}{%
    \hphantom{\rule{0.05em}{0ex}}%
    #2%
    \hphantom{\rule{0.05em}{0ex}}%
    \vphantom{\rule[-0.5ex]{0em}{2.5ex}}%
  }%
}%
\definecolor{coolblack}{rgb}{0.0, 0.18, 0.39}
\newcommand\heartcard{\card{white}{\color{red}{♥}}}
\newcommand\clubcard{\card{white}{\color{coolblack}{♣}}}
\newcommand\commitedcard{\card{gray!20}{\hphantom{\rule{0.28em}{0ex}}?\hphantom{\rule{0.28em}{0ex}}}}%}}
% \def\yescards{\heartcard\,\heartcard\,\clubcard}
% \def\nocards{\heartcard\,\clubcard\,\clubcard}
% \def\threecommitedcards{\commitedcard\,\commitedcard\,\commitedcard}
% \def\threeheartcards{\heartcard\,\heartcard\,\heartcard}
% \def\threeclubcards{\clubcard\,\clubcard\,\clubcard}

\newcommand\ce[1]{%
  \coloremojiucs{#1}
}

\newcommand*{\lstitem}[1]{
  \setbox0\hbox{\lstinline{#1}}
  \item[\usebox0]
}

\presetkeys{todonotes}{inline, noinlinepar}{}

\renewcommand{\arraystretch}{1.2}
\newcolumntype{Y}{>{\centering\arraybackslash}X}

\title[Datatype Generic Programming with Scala 3]{%
  Datatype Generic Programming \\
  with Scala 3
}
\subtitle{}
\author[Yoshimura Hikaru]{%
  \textsc{Yoshimura} Hikaru
}
\Email{hikaru\_yoshimura@r.recruit.co.jp}
\date[April 15-16, 2023]{%
  \oldstylenums{April 15-16, 2023}
}
\Conference[ScalaMatsuri 2023]{Sponsor sessions of ScalaMatsuri 2023}
\institute[\InsertEmail]{Recruit Co., Ltd}
\GitHub{y-yu/scalamatsuri2023}

\newcommand{\facesize}{1cm}
\newcommand\alicecallout[2]{
  \simplecallout[{\includegraphics[width=\facesize]{./img/alice_face.png}}]{#1}{cyan!10}{#2}
}
\newcommand\bobcallout[2]{
  \simplecallout[{\includegraphics[width=\facesize]{./img/bob_face.png}}]{#1}{orange!10}{#2}
}

\begin{document}

\frame{%
  \maketitle
  \note{%
    \vspace*{\fill}
    \centering
    {\LARGE Datatype Generic Programming with Scala 3}
    \vspace{2ex}
    
    \begin{itemize}
      \item こっちのページは日本語の説明となります。
    \end{itemize}
    \vspace*{\fill}
  }%
}

\section{Introduction}

\begin{frame}
  \frametitle{Table of contents}

  \tableofcontents
\end{frame}

\begin{frame}
  \frametitle{Who am I?}
  
  \begin{columns}
    \begin{column}{0.3\textwidth}
      \begin{center}
        \begin{figure}
          \includegraphics[width=0.95\textwidth]{img/bird2x.png}
        \end{figure}
      \end{center}
 
      \begin{table}[h]
        \begin{tabular}{ll}
          Twitter & \href{https://twitter.com/\_yyu\_}{@\_yyu\_} \\
          Qiita &  \href{https://qiita.com/yyu}{yyu} \\
          GitHub &  \href{https://github.com/y-yu}{y-yu} \\
        \end{tabular}
      \end{table}
    \end{column}
    \begin{column}{0.7\textwidth}
      \begin{itemize}
        \item Recruit Co., Ltd.
        \begin{itemize}
          \item StudySapuri ENGLISH server side
        \end{itemize}

        \item Quantum Information \& Algorithms

        \item Cryptography \& Security
        
        \item Programming \& {\LaTeX} typesetting
        \begin{itemize}
          \item Scala, Rust, Go, Swift
        \end{itemize}
      \end{itemize}
    \end{column}
  \end{columns}
\end{frame}

\begin{frame}[fragile]
  \frametitle{\lstinline|TestObject|: generating fixtures for unit tests}

  \begin{itemize}
    \item We uses \lstinline|TestObject| on our product,
    which is a utility to generate dummy objects(as known as \emph{fixtures}) for unit tests.
\begin{lstlisting}[style=scala]
case class StudySapuriSession(
  /* very complicated! */
)
val dummyData = TestObject.get[StudySapuriSession]
\end{lstlisting}

    \item Type class \lstinline|TestObject[A]| provides us to the way for generating some value of \lstinline|A|.
\begin{lstlisting}[style=scala]
trait TestObject[A] {
  def generate: State[Int, A]
}
\end{lstlisting}

    \item In naive way, we have to define too many \lstinline|TestObject| implicit instances for
    every types used in our product, but it's not possible and reasonable.
  \end{itemize}
\end{frame}

\begin{frame}
  \frametitle{\lstinline|TestObject|: generating fixtures for unit tests}

  \begin{itemize}
    \item The many \lstinline|TestObject| instances can be provided by
    \emph{datatype generic programming}, not manually.
    \begin{itemize}
      \item We can get \lstinline|dummyData: StudySapuriSession| easily once we define
      \lstinline|TestObject| instances for primitive types,
      \item Then datatype generic programming generates the other instances for our
      defined data structures(= case objects).
    \end{itemize}

    \item In this talk I'll explain what the differences of datatype generic programming
    between Scala 2 and 3.
  \end{itemize}

  \note{
    \begin{itemize}
      \item TODO: タイトルどうにかする
    \end{itemize}
  }
\end{frame}

\begin{frame}[fragile]
  \frametitle{What is datatype generic programming}

  \begin{itemize}
    \item Datatype generic programming is the way of meta-programming.

    \item If we would use raw macros, we can create a new syntax like S-expression in Scala source,
    but we don't want it.
    %\begin{itemize}
    %  \item I think template meta-programming in C++ allows us to prevent creating a new syntax.
    %\end{itemize}

    \item Every data structure can be classified either ``tuple'' like or ``enum'' like:
    \begin{columns}
      \begin{column}{0.4\textwidth}
\begin{lstlisting}[style=scala, caption=tuple like]
case class TupleLike(
  field1: Int, field2: String
)
\end{lstlisting}
      \end{column}
      \begin{column}{0.6\textwidth}
\begin{lstlisting}[style=scala, caption=enum like]
sealed trait EnumLike
case class Pattern1(v: Int)    extends EnumLike
case class Pattern2(v: String) extends EnumLike
\end{lstlisting}
      \end{column}
    \end{columns}
    \begin{itemize}
      \item \lstinline|TupleLike| requires both two values of \lstinline|Int| and \lstinline|String|,
      on the other hand \lstinline|EnumLike| requires either \lstinline|Int| value or \lstinline|String| value.
    \end{itemize}

    \item Datatype generic programming provides us to such abstraction
    for arbitrary types.
  \end{itemize}

  \note{
    \begin{itemize}
      \item TODO: ``Datatype generic programming provides us to such abstraction
      for arbitrary types.''これもうちょっといい感じにする
      \begin{itemize}
        \item タプルとかenumのように抽象化する、みたいなニュアンス加える
      \end{itemize}
    \end{itemize}
  }
\end{frame}

\begin{frame}
  \frametitle{What is datatype generic programming}
  
  \begin{itemize}
    \item Almost all meta-programming can be done by such datatype abstraction
    and ad-hoc polymorphism, without probability of creating a new syntax.
    \begin{enumerate}
      \item First convert user defined data structures(= case objects) to tuple or enum like using datatype generic programming
      \item Then find some implicit instances based on the types
      \item Finally convert the derived instance for tuple or enum like to one for the original data type.
    \end{enumerate}

    \item We uses ``shapeless''\cite{shapeless_github} for datatype generic programming in Scala 2.
  \end{itemize}

  \note{
    \begin{itemize}
      \item TODO: ユーザー定義データ構造とdatatype generic programmingとad-hoc多相の関係を図にする
    \end{itemize}
  }
\end{frame}

\begin{frame}
  \frametitle{Macro compatibility between Scala 2 and 3}

  \begin{itemize}
    \item Likewise on our product ``StudySapuri ENGLISH'', 
    we are compiling both of Scala 2 and 3 for almost every code.

    \item There is no compatibility of \emph{macros} between Scala 2 and 3 \ce{:innocent:}

    \item It follows that \lstinline|TestObject| implemented on Scala 2 won't work well on Scala 3.

    \item shapeless for Scala 3 is being developed but unfortunately
     it's no compatibility shapeless for Scala 2 \ce{:innocent:}

    \item Eventually we(mainly ScalaNinja) implemented \lstinline|TestObject| for Scala 3
  \end{itemize}
\end{frame}

\section{Datatype generic programming in Scala 3}

\begin{frame}[fragile]
  \frametitle{Datatype generic programming in Scala 3}

  \begin{itemize}
    \item Scala 3 supports datatype generic programming initially like follows:
    % using
    % \lstinline|summonFrom|, \lstinline|Mirror.ProductOf| and \lstinline|Mirror.SumOf|.
    \begin{columns}
      \begin{column}{0.4\textwidth}
\begin{lstlisting}[style=scala]
import scala.compiletime.*
import scala.deriving.*
case class TupleLike(
  field1: Int, field2: String
)
\end{lstlisting}
      \end{column}
      \begin{column}{0.6\textwidth}
\begin{lstlisting}[style=scala]
scala> Tuple.fromProductTyped(TupleLike(1, "a"))
val res0: (Int, String) = (1,a)

scala> summon[Mirror.ProductOf[TupleLike]].fromProduct(res0)
val res1: TupleLike = TupleLike(1,a)
\end{lstlisting}
      \end{column}
    \end{columns}
    \begin{itemize}
      \item We can use some functions to convert
      case objects from/to tuple like without any libraries.
    \end{itemize}

  \end{itemize}
\end{frame}

\begin{frame}[fragile]
  \frametitle{\lstinline|TestObject| implementation on Scala 3}

  \begin{itemize}
    \item We'll define \lstinline|derive| method such like:
\begin{lstlisting}[style=scala]
inline implicit def derive[A]: TestObject[A]
\end{lstlisting}   
    \begin{itemize}
      \item \lstinline|derive| provides \lstinline|TestObject| instances for all \lstinline|A|.
    \end{itemize}

    \begin{enumerate}
      \item Create the instances list of \lstinline|TestObject| using \lstinline|erasedValue|.
    \end{enumerate}
  \end{itemize}
\end{frame}


\begin{frame}[fragile]
  \frametitle{\ballcircle{1} Input type instance has been defined or not}

  \begin{itemize}
    \item \lstinline|summonFrom| searches the \lstinline|TestObject| instance for type \lstinline|A|
\begin{lstlisting}[style=scala]
inline implicit def derive[A]: TestObject[A] =
  summonFrom {
    case x: TestObject[A] =>
      x
    case _ =>
      create[A] // we'll define next page!
  }
\end{lstlisting}

    \item If Scala 3 can find \lstinline|TestObject[A]| instance named by \lstinline|x|, then
    it's unnecessary to define the instance so returns \lstinline|x|.
  \end{itemize}
\end{frame}

\begin{frame}[fragile]
  \frametitle{\ballcircle{2} Input type is either product or sum}

  \begin{itemize}
    \item Otherwise there is no instance yet \lstinline|create| splits \lstinline|A| to
    \lstinline|ProductOf| or \lstinline|SumOf|.
\begin{lstlisting}[style=scala]
inline final def create[A]: DeterministicTestObject[A] =
  summonFrom {
    case _: Mirror.ProductOf[A] =>
      deriveProduct[A]
    case _: Mirror.SumOf[A] =>
      deriveSum[A]
  }
\end{lstlisting}

    \item If it goes to case:
    \begin{description}
      \item[\texttt{ProductOf}] the input type \lstinline|A| is
      tuple like (for example case objects).
      
      \item[\texttt{SumOf}] input type \lstinline|A| is enum like
      \lstinline|sealed trait|.
    \end{description}
  \end{itemize}
\end{frame}

\begin{frame}[fragile]
  \frametitle{\ballcircle{3a} \lstinline|Product| case}
  
  \begin{itemize}
    \item We have to get all instances for types being contained in \lstinline|A|.
    \begin{columns}
      \begin{column}{0.6\textwidth}
        \begin{itemize}
          \item For example \lstinline|TupleLike| contains two type: \lstinline|Int| and \lstinline|String|.

          \item We need the both instances of \lstinline|TestObject[Int]| and \lstinline|TestObject[String]|.
        \end{itemize}
      \end{column}
      \begin{column}{0.3\textwidth}
\begin{lstlisting}[style=scala]
case class TupleLike(
  field1: Int, field2: String
)
\end{lstlisting}
      \end{column}
    \end{columns}

    \item \lstinline|erasedValue| allows us to search and create all instances recursively.
\begin{lstlisting}[style=scala]
inline def deriveRec[T <: Tuple]: List[TestObject[?]] =
  inline erasedValue[T] match {
    case _: EmptyTuple =>
      Nil
    case _: (t *: ts) =>
      derive[t] :: deriveRec[ts]
  }
\end{lstlisting}
  \end{itemize}
\end{frame}

\begin{frame}[fragile]
  \frametitle{\ballcircle{3a} \lstinline|Product| case}
  
  \begin{itemize}
    \item Using \lstinline|deriveRec|, we finally define \lstinline|TestObject| instance for \lstinline|A|
\begin{lstlisting}[style=scala]
inline def deriveProduct[A](using a: Mirror.ProductOf[A]): TestObject[A] = {
  val xs = deriveRec[a.MirroredElemTypes]
  new TestObject[A] {
    def generate: IntState[A] =
      for {
        values <- ApplicativeIntState
          .traverse(xs)(_.generate.widen[Any])
        _ <- modify[Int](_ + 1)
      } yield a.fromProduct(new SeqProduct(values))
  }
}
\end{lstlisting}
  \end{itemize}
\end{frame}

\begin{frame}[fragile]
  \frametitle{\ballcircle{3b} \lstinline|Sum| case}

  \begin{itemize}
    \item 
\begin{lstlisting}[style=scala],
  final def sumImpl[A](values: List[DeterministicTestObject[?]]): DeterministicTestObject[A] =
    new DeterministicTestObject[A] {
      def generate: IntState[A] = {
        for {
          allResults <- ApplicativeIntState
            .traverse(values)(_.generate.widen[Any])
          l = allResults.minBy(_.getClass.getName)
\end{lstlisting}
  \end{itemize}
\end{frame}

% \section{奢り・割勘問題}

% \begin{frame}
%   \frametitle{奢り・割勘問題}

%   \begin{figure}[h]
%     \includegraphics[width=0.65\textwidth]{./img/twitter.png}\cite{Y_N_Hoshi}
%   \end{figure}
% \end{frame}

% \begin{frame}
%   \frametitle{奢り・割勘問題}

%   \begin{shadequote}[r]{}
%     \begin{center}
%       アリスとボブの飲食費について下記のいずれにするか決定する問題
%       \begin{enumerate}
%         \item ボブが全額を奢る
%         \item 割勘とする
%       \end{enumerate}
%     \end{center}
%   \end{shadequote}

%   \begin{columns}
%     \begin{column}{0.5\textwidth}
%       \centering
%       \emph{アリス(Alice)}

%       \begin{figure}[h]
%         \includegraphics[height=0.4\textheight]{img/alice.png}
%       \end{figure}
%     \end{column}
   
%     \begin{column}{0.5\textwidth}
%       \centering
%       \emph{ボブ(Bob)}

%       \begin{figure}[h]
%         \includegraphics[height=0.4\textheight]{img/bob.png}
%       \end{figure}
%     \end{column}
%   \end{columns}
% \end{frame}

% \section{従来手法}

% \begin{frame}
%   \frametitle{従来手法\ballcircle{1} --- コイントス}

%   \vspace{-0.2cm}
%   \pause
%   \begin{columns}
%     \begin{column}{0.5\textwidth}
%       \begin{enumerate}
%         \item 2人でコイントスを行う

%         \pause
%         \item 表\ref{tbl:coin_meaning}に基づいて決定する
%       \end{enumerate}

%       \begin{table}[h]
%         \caption{コインの意味}
%         \label{tbl:coin_meaning}
%         \begin{tabularx}{0.9\textwidth}{@{}| Y | Y |@{}}
%           \hline
%           コインの結果 & 意味 \\ \hline
%           表 & ボブの奢り \\ \hline
%           裏 & 割勘 \\ \hline
%         \end{tabularx}
%       \end{table}
%     \end{column}
%     \begin{column}{0.5\textwidth}
%       \pause
%       \bobcallout{+}{2人の意見が何も反映されないため\\
%         プライバシーは完全\ce{:person_gesturing_ok:}%
%       }
    
%       \pause
%       \alicecallout{-}{しかしゲーム性は全くない}
%     \end{column}
%   \end{columns}
% \end{frame}

% \begin{frame}
%   \frametitle{従来手法\ballcircle{1} --- コイントス}

%   \begin{columns}
%     \begin{column}{0.6\textwidth}
%       \alicecallout{+}{そもそもこのゲームはアリスが有利\ce{:smiling_imp:}}

%       \pause
%       \bobcallout{-}{%
%         そもそも不公平なゲームなので、\\
%         50:50ではおもしろくない
%       }
%     \end{column}
%     \begin{column}{0.4\textwidth}
%       \begin{itemize}
%         \item ボブが支払うのは安目で50\%高目で100\%だが、
%         一方でアリスは安目0\%高目でも50\%
%       \end{itemize}
%     \end{column}
%   \end{columns}
% \end{frame}

% \begin{frame}
%   \frametitle{従来手法\ballcircle{2} --- 公平な第三者を用いたAND計算}

%   \pause
%   \begin{enumerate}
%     \item<+-> アリスとボブは公平な第三者チャーリーに$\text{\textbf{希望}} \in \{\text{奢り}, \text{割勘}\}$を渡す
%     \item<+-> チャーリーはアリス・ボブの希望を次の表\ref{tbl:and_table}に基づいてAND演算する
%     \item<+-> チャーリーが結果を2人に通知する
%   \end{enumerate}

%   \uncover<+(-2)->{
%     \begin{table}[h]
%       \caption{奢り・割勘AND演算}
%       \label{tbl:and_table}
%       \begin{tabularx}{0.7\textwidth}{@{}| Y | Y || Y |@{}}
%         \hline
%         アリス & ボブ & 結果 \\ \hline
%         割勘  & 割勘 & 割勘 \\ \hline
%         割勘  & 奢り & 割勘 \\ \hline
%         奢り  & 割勘 & 割勘 \\ \hline
%         奢り  & 奢り & 奢り \\ \hline
%       \end{tabularx}
%     \end{table}
%   }

%   \begin{itemize}
%     \item<+(-1)-> ボブは期待値は$\frac{5}{8}$(62.5\%)の支払い、
%     アリスは期待値$\frac{3}{8}$(37.5\%)の支払い
%   \end{itemize}
% \end{frame}

% \begin{frame}
%   \frametitle{従来手法\ballcircle{2} --- 公平な第三者を用いたAND計算}
  
%   \begin{columns}
%     \begin{column}{0.6\textwidth}
%       \bobcallout{+}{そもそもチャーリーが信頼できるのか?
%       }

%       \pause
%       \alicecallout{-}{\noindent
%         次のケース\ce{:point_down:}で\textbf{情報リーク}が生じる!\\
%         \hbox to 0.5\textwidth{%
%           \vbox{%
%             \begin{enumerate}
%               \item アリスが奢り、ボブは割勘を希望
%               \item アリスが割勘、ボブは奢りを希望\label{enum:bob_detect}
%               \item アリスが奢り、ボブは奢りを希望
%             \end{enumerate}
%           }
%         }
%       }
%     \end{column}
%     \begin{column}{0.4\textwidth}
%       \begin{itemize}
%         \item AND計算なので片方の入力と結果から、残りの入力を逆算できる場合がある
%       \end{itemize}
%     \end{column}
%   \end{columns}
% \end{frame}

% \begin{frame}
%   \frametitle{従来手法\ballcircle{2} --- 公平な第三者を用いたAND計算}

%   \begin{columns}
%     \begin{column}{0.6\textwidth}
%       \alicecallout{+}{希望が場合によっては流出するのは\\
%         ゲーム性とみなすことができそう!\ce{:smiling_imp:}
%       }

%       \pause
%       \bobcallout{-}{しかしこの方法では\\
%         アリスの希望は絶対にリークしない\ce{:rage:}%
%       }
%     \end{column}
%     \begin{column}{0.4\textwidth}
%       \begin{itemize}
%         \item たとえば「アリスが奢り・ボブは割勘を希望」のとき、
%         結果は割勘となる
        
%         \item アリスは奢ってもらえないが、ボブのケチさを知ることができる

%         \item しかしアリスのがめつさ情報がリークすることはない

%         \pause
%         \item なぜなら\textbf{衝突}したときに
%         ``奢り''になるケースがないから
%       \end{itemize}
%     \end{column}
%   \end{columns}
% \end{frame}

% \begin{frame}
%   \frametitle{2人の希望の衝突}

%   \pause
%   \begin{columns}
%     \begin{column}{0.6\textwidth}
%       \uncover<+-> {
%         \alicecallout{+}{%
%           2人の希望が一致したケースなら\\
%           AND計算で問題ない%
%         }
%       }

%       \uncover<+->{
%         \bobcallout{-}{%
%           一方で衝突(\textit{conflict})した\\
%           ケースは考える必要がある
%         }
%       }
%     \end{column}
%     \uncover<1->{
%       \begin{column}{0.4\textwidth}
%           \begin{table}[h]
%             \caption{現状のコンセンサス}
%             \label{tbl:conflict}
%             \begin{tabularx}{0.9\textwidth}{@{}| Y | Y | X @{}}
%               \cline{1-2}
%               アリス & ボブ & \\ \cline{1-2}
%               割勘  & 割勘 & \multirow{2}{\hsize}{$\bigg\}$\Large\ce{:person_gesturing_ok:}}\\ \cline{1-2}
%               奢り  & 奢り &  \\ \cline{1-2}
%               割勘  & 奢り & \multirow{2}{\hsize}{$\bigg\}$\Large\ce{:person_gesturing_no:}} \\ \cline{1-2}
%               奢り  & 割勘 &   \\ \cline{1-2}
%             \end{tabularx}
%           \end{table}
%       \end{column}
%     }
%   \end{columns}
% \end{frame}

% \section{提案手法}

% \begin{frame}
%   \centering
%   \begin{columns}
%     \begin{column}{0.2\textwidth}
%       \begin{minipage}[t][.8\textheight][t]{\textwidth}
%         \tableofcontents[currentsection]
%       \end{minipage}
%     \end{column}
%     \begin{column}{0.8\textwidth}
%       \centering
%       \begin{columns}
%         \begin{column}{0.45\columnwidth}
%           \alicecallout{-}{\huge コイントス}
%         \end{column}
%         \begin{column}{0.45\columnwidth}
%           \bobcallout{+}{\huge AND計算}
%         \end{column}
%       \end{columns}
%       {\centering\huge%
%         $\underbrace{\hphantom{\rule{\columnwidth}{0ex}}}_{}$\\
%         \mcfamily\rmfamily\bfseries ハイブリッドプロトコル%
%       }
%     \end{column}
%   \end{columns}
% \end{frame}

% \begin{frame}
%   \frametitle{コイントスとAND計算のハイブリッドプロトコル}

%   \begin{columns}
%     \begin{column}{0.6\textwidth}
%       \uncover<+-> {
%         \bobcallout{+}{%
%           表\ref{tbl:choice_and_info_leak}のようにランダム\footnotemark[3] を導入したうえで、\\
%           不本意な結果を強いられた側だけが\\
%           相手の希望を得るというのはどうか?
%         }
%       }

%       \footnotetext[3]{コイントス同様に奢り・割勘それぞれ$\frac{1}{2}$の確率}
%     \end{column}
%     \uncover<1->{
%       \begin{column}{0.4\textwidth}
%           \begin{table}[h]
%             \caption{奢り・割勘と情報リーク}
%             \label{tbl:choice_and_info_leak}
%             \begin{tabular}{|c|c|c|c|}
%               \hline
%               アリス & ボブ & 結果   & 情報 \\ \hline
%               割勘  & 割勘 & 割勘   & \dagger \\ \hline
%               割勘  & 奢り & ランダム & \ddagger \\ \hline
%               奢り  & 割勘 & ランダム & \ddagger  \\ \hline
%               奢り  & 奢り & 奢り   & \dagger \\ \hline
%             \end{tabular}

%             \footnotetext[1]{お互いの希望はいずれもリークしない} 
%             \footnotetext[2]{結果が奢りの場合はアリスの希望がボブへ、結果が割勘の場合はボブの希望がアリスへリークする}
%           \end{table}
%       \end{column}
%     }
%   \end{columns}
% \end{frame}

% \begin{frame}
%   \frametitle{手順}

%   このプロトコルはチャーリー(\emph{trusted third party})\textbf{なし}で達成できる

%   \pause
%   \begin{columns}
%     \begin{column}{0.6\textwidth}
%       \begin{enumerate}
%         \item アリス・ボブに2枚のカード\heartcard,\clubcard を配る\footnote{%
%           これらのカードはトランプのようにいずれも裏が\commitedcard となっており、
%           裏向きになった状態でどちらのカードなのか特定することができない
%         }
%         \item アリス・ボブは表\ref{tbl:card_meaning}に従って
%         希望を裏向き\commitedcard にして提出する\label{enum:cards_commited}

%         \item \ballref{enum:cards_commited}で提出されたカードをシャッフルする
        
%         \item どちらか1枚をドローして表向きにする \label{enum:result}
%       \end{enumerate}

%       \ballref{enum:result}のカードを表\ref{tbl:card_meaning}に対応させてプロトコルの結果とする
%     \end{column}
%     \begin{column}{0.4\textwidth}
%       \begin{table}[h]
%         \caption{カードの意味}
%         \label{tbl:card_meaning}
%         \begin{tabularx}{0.9\textwidth}{@{}| Y | Y |@{}}
%           \hline
%           カード & 意味 \\ \hline
%           \heartcard & ボブの奢り \\ \hline
%           \clubcard & 割勘 \\ \hline
%         \end{tabularx}
%       \end{table}
%     \end{column}
%   \end{columns}
% \end{frame}

% \begin{frame}
%   \frametitle{ケーススタディ\ballcircle{1} --- 2人の希望が一致}

%   \pause
%   \begin{itemize}
%     \item<+-> 2人の希望が一致しているので次のようなケース
%       \begin{columns}
%         \begin{column}{0.5\textwidth}
%           \alicecallout{+}{\heartcard}
%         \end{column}
%         \begin{column}{0.5\textwidth}
%           \bobcallout{-}{\heartcard}
%         \end{column}
%       \end{columns}

%     \item<+-> これらをシャッフルして1枚選んだときは必ず\heartcard となる

%     \item<+-> そしてこのときアリス・ボブは相手のカードについて
%     \begin{itemize}
%       \item 両方とも\heartcard だったのか
%       \item 相手は\clubcard だったがランダムで\heartcard が選ばれたのか
%     \end{itemize}
%     \ce{:point_up:}のどちらなのか分からず、情報リークはない
%   \end{itemize}
% \end{frame}

% \begin{frame}
%   \frametitle{ケーススタディ\ballcircle{2} --- 2人の希望が衝突}

%   \begin{itemize}
%     \item<+-> 2人の希望が衝突しているので次のようなケース
%       \begin{columns}
%         \begin{column}{0.5\textwidth}
%           \alicecallout{+}{\heartcard}
%         \end{column}
%         \begin{column}{0.5\textwidth}
%           \bobcallout{-}{\clubcard}
%         \end{column}
%       \end{columns}

%     \item<+-> これらをシャッフルしてランダムに選べば、結果は\heartcard,\clubcard それぞれ$\frac{1}{2}$の確率になる
%     \begin{description}
%       \item[結果が\heartcard]<+->\mbox{}\\
%       \begin{itemize}
%         \item アリスの希望通りとなるが、結果がボブの希望通りかランダムか不明
%         \item ボブは衝突してアリスの希望\heartcard になったと特定
%       \end{itemize}

%       \item[結果が\clubcard]<+-> 同様
%     \end{description}
%   \end{itemize}
% \end{frame}

% \begin{frame}
%   \frametitle{コイントスとAND計算のハイブリッドプロトコル}

%   \bobcallout{+}{%
%     期待値的にはボブが不公平なままだが、\\
%     もし不本意に奢った場合はアリスのがめつさが分かる
%   }

%   \pause
%   \alicecallout{-}{%
%     このときアリスはボブの奢りが本意か\\
%     不本意か分からないが、奢られを得る
%   }
% \end{frame}

% \begin{frame}
%   \frametitle{コイントスとAND計算のハイブリッドプロトコル}

%   \begin{columns}
%     \begin{column}{0.6\textwidth}
%       \uncover<+->{
%         \alicecallout{+}{%
%           逆にアリスが不本意に\\
%           割勘となってしまった場合、\\
%           ボブの希望は割勘だと特定する
%         }
%       }

%       \uncover<+->{
%         \bobcallout{-}{%
%           しかしこのときボブは\\
%           アリスの希望が分からない
%         }
%       }
%     \end{column}
%     \begin{column}{0.4\textwidth}
%       \uncover<+(-2)->{
%         \begin{itemize}
%           \item アリスが不本意に割勘となった場合、
%           アリスは奢りを希望していたがボブは割勘を希望しており、
%           ランダムで割勘となった

%           \item<2-> このように希望通りになった側は相手の希望が分からず、
%           希望通りにならかった側は相手の希望を知ることができる
%         \end{itemize}
%       }
%     \end{column}
%   \end{columns}
% \end{frame}

\section{Conclusion}

\begin{frame}
  \frametitle{Conclusion}

  \pause
  \begin{itemize}
    \item<+-> S
  \end{itemize}
\end{frame}

\section*{References}
\begin{frame}%[allowframebreaks]
  \frametitle{References}
  \nocite{*}
  \bibliographystyle{junsrt_url}
  \bibliography{ref}
\end{frame}

\begin{frame}
  \centering
  {\Huge Thank you for the attention!}
\end{frame}

\end{document}
