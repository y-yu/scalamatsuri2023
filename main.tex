% for notes environment
\usepackage{xsavebox}
\usepackage{hyperref}
\usepackage{graphicx}
\usepackage{luatexja}
\usepackage[hiragino-pro,deluxe,nfssonly,jis2004]{luatexja-preset}
\usepackage{fontspec}
\usepackage{epigraph}
\usepackage{etoolbox}
\usepackage{tikz}
\usepackage{framed}
\usepackage{mathtools}
\usepackage{listings}
\usepackage{libertine}
\usepackage[libertine]{newtxmath}
\usepackage{bxcoloremoji}
\usepackage{xcolor}
\usepackage{diagbox}
\usepackage{caption}
\usepackage{appendixnumberbeamer}
\usepackage{multirow}
\usepackage{xpatch}
\usepackage{multicol}
\usepackage{tabularx}
\usepackage{bussproofs}

\newenvironment{scaledprooftree}[1]%
  {\gdef\scalefactor{#1}\begin{center}\proofSkipAmount \leavevmode}%
  {\scalebox{\scalefactor}{\DisplayProof}\proofSkipAmount \end{center} }

\usetikzlibrary{fit}

\setmonofont{CMU Typewriter Text}

\definecolor{links}{HTML}{2A1B81}
\hypersetup{colorlinks,linkcolor=,urlcolor=links}

\usetheme{Boadilla}
\usecolortheme{seahorse}
% \usefonttheme{serif}


\xpatchcmd{\itemize}
  {\def\makelabel}
  {\ifnum\@itemdepth=1\relax
     \setlength\itemsep{1.2ex}% separation for first level
   \else
     \ifnum\@itemdepth=2\relax
       \setlength\itemsep{0.8ex}% separation for second level
       \setlength\topsep{1.2ex}
     \else
       \ifnum\@itemdepth=3\relax
         \setlength\itemsep{0.05ex}% separation for third level
         \setlength\topsep{0.8ex}
   \fi\fi\fi\def\makelabel
  }
 {}
 {}

\setbeamercolor{page number in head/foot}{bg=blue!10}
\setbeamertemplate{footline}{%
  \leavevmode%
  \hbox{%
    \begin{beamercolorbox}[wd=.3\paperwidth,ht=2.25ex,dp=1ex,center]{author in head/foot}%
      \usebeamerfont{author in head/foot}\insertshortauthor\hspace*{1ex}(\insertshortinstitute)
    \end{beamercolorbox}%
    \begin{beamercolorbox}[wd=.3\paperwidth,ht=2.25ex,dp=1ex,center]{title in head/foot}%
      \usebeamerfont{title in head/foot}\insertshorttitle
    \end{beamercolorbox}%
    \begin{beamercolorbox}[wd=.3\paperwidth,ht=2.25ex,dp=1ex,center]{date in head/foot}%
      \insertshortdate{} @ \InsertConferenceShort
    \end{beamercolorbox}%
    \begin{beamercolorbox}[wd=.1\paperwidth,ht=2.25ex,dp=1ex,center]{page number in head/foot}%
      \insertframenumber{} / \inserttotalframenumber\hspace*{1ex}
    \end{beamercolorbox}}%
  \vskip0pt%
}

\beamertemplatenavigationsymbolsempty

\setbeamertemplate{bibliography item}{\insertbiblabel}
\setbeamersize{description width=1cm}
\setbeamertemplate{items}[circle]
\setbeamertemplate{section in toc}[circle]
\setbeamertemplate{subsection in toc}{%
  \leavevmode\leftskip=2em
  {%
    \usebeamerfont*{itemize item}%
    \usebeamercolor{subsection number projected}%
    \color{bg}%
    \raise1.25pt\hbox{\donotcoloroutermaths$\bullet$}}%
  \hskip1.5ex\inserttocsubsection\par}

% Definitions for the title page
\newcommand*{\GitHub}[1]{%
  \gdef\InsertGitHub{#1}%
}
\newcommand*{\Email}[1]{%
  \gdef\InsertEmail{\href{mailto:#1}{#1}}%
}
\newcommand{\ConferenceImpl}[2][]{%
  \gdef\InsertConferenceShort{#1}%
  \gdef\InsertConference{#2}%
}
\makeatletter
\newcommand\Conference{\@dblarg\ConferenceImpl}
\makeatother
\setbeamerfont{title}{size=\huge, series=\bfseries, family=\mcfamily\rmfamily}
\setbeamercolor{title}{bg=white}
\setbeamerfont{subtitle}{size=\large, series=\mdseries, family=\gtfamily\sffamily}
\setbeamerfont{email}{size=\scriptsize, family=\ttfamily}
\setbeamercolor{email}{bg=white}
\setbeamerfont{date}{shape=\itshape, family=\rmfamily}
\setbeamerfont{vc}{size=\scriptsize, family=\ttfamily}
\setbeamercolor{vc}{bg=white}

\renewcommand{\figurename}{Fig}

\input{vc.tex}

\setbeamertemplate{title page}
{%
  \vbox{}
  \vfill
  \begingroup
    \centering
    \hrulefill\par%
    \vskip1ex\par%
    \begin{beamercolorbox}[sep=0pt,center,shadow=false,rounded=true]{title}
      \vfill
      \usebeamerfont{title}\inserttitle\par%
      \ifx\insertsubtitle\@empty%
      \else%
        \vskip0.5ex%
        {\usebeamerfont{subtitle}\usebeamercolor[fg]{subtitle}\insertsubtitle\par}%
      \fi%
      \vfill  
    \end{beamercolorbox}%
    \hrulefill\par%
    \vskip2ex%
    \begin{beamercolorbox}[sep=0pt,center,shadow=false,rounded=true]{author}
      \usebeamerfont{author}\insertauthor
    \end{beamercolorbox}
    \begin{beamercolorbox}[sep=0pt,center,shadow=false,rounded=true]{email}
      \usebeamerfont{email}\InsertEmail
    \end{beamercolorbox}
    \vskip0.1ex
    \begin{beamercolorbox}[sep=5pt,center,shadow=false,rounded=true]{institute}
      \usebeamerfont{institute}\insertinstitute
    \end{beamercolorbox}
    \begin{beamercolorbox}[sep=5pt,center,shadow=false,rounded=true]{date}
      \usebeamerfont{date}\insertdate \normalfont @ \InsertConference
    \end{beamercolorbox}
    \begin{beamercolorbox}[sep=0pt,center,shadow=false,rounded=true]{vc}
      \usebeamerfont{vc}
      \url{https://github.com/\InsertGitHub} (\texttt{\GITAbrHash})
    \end{beamercolorbox}
    % {\centering
    %   \href{https://creativecommons.org/licenses/by-nc/4.0/}{%
    %     \includegraphics[width=0.1\textwidth]{img/by-nc.pdf}%
    %   }%
    % }
    {\usebeamercolor[fg]{titlegraphic}\inserttitlegraphic\par}
  \endgroup
  \vfill
}
\setbeamertemplate{blocks}[rounded][shadow=false]

% ============ ここを消すとNote消える ================
\mode<handout>{%
  \usepackage{pgfpages}
  \setbeameroption{show notes on second screen=right}
  \setbeamertemplate{note page}{%
    \vspace{2ex}\insertnote%
  }
}
% ============ ここを消すとNote消える ================


\renewcommand{\kanjifamilydefault}{\gtdefault}

\setbeamertemplate{caption}[numbered]
\resetcounteronoverlays{lstlisting}
\definecolor{bluegray}{rgb}{0.4, 0.6, 0.8}
\DeclareCaptionFormat{listing}{{\color{bluegray}\lstlistingname}#2#3}
\captionsetup[lstlisting]{format=listing, font={footnotesize}}
\captionsetup[figure]{name={Figure}}
\captionsetup[table]{name={Table}}
\setbeamerfont{footnote}{size=\scriptsize}

\setmonofont[Ligatures=TeX]{CMU Typewriter Text}

\setbeamertemplate{items}[circle]

\newfontfamily\quotefont[Ligatures=TeX]{Linux Libertine O} % selects Libertine as the quote font

\newcommand*\quotesize{60} % if quote size changes, need a way to make shifts relative
% Make commands for the quotes
\newcommand*{\openquote}
   {\tikz[remember picture,overlay,xshift=0em,yshift=-3ex]
   \node (OQ) {\quotefont\fontsize{\quotesize}{\quotesize}\selectfont``};\kern0pt}

\newcommand*{\closequote}[1]
  {\tikz[remember picture,overlay,xshift=1ex,yshift={#1}]
   \node (CQ) {\quotefont\fontsize{\quotesize}{\quotesize}\selectfont''};}

\newcommand*\shadedauthorformat{\emph} % define format for the author argument

% Now a command to allow left, right and centre alignment of the author
\newcommand*\authoralign[1]{%
  \if#1l
    \def\authorfill{}\def\quotefill{\hfill}
  \else
    \if#1r
      \def\authorfill{\hfill}\def\quotefill{}
    \else
      \if#1c
        \gdef\authorfill{\hfill}\def\quotefill{\hfill}
      \else\typeout{Invalid option}
      \fi
    \fi
  \fi}
% wrap everything in its own environment which takes one argument (author) and one optional argument
% specifying the alignment [l, r or c]
%
\newenvironment{shadequote}[2][l]%
{\hspace{0.5ex}
\authoralign{#1}
\ifblank{#2}
   {\def\shadequoteauthor{}\def\yshift{-1ex}\def\quotefill{\hfill}}
   {\def\shadequoteauthor{\par\authorfill\shadedauthorformat{#2}}\def\yshift{3ex}}
\begin{quote}\normalfont\openquote}
{\shadequoteauthor\quotefill\closequote{\yshift}\end{quote}}

\input{./lib/footnotemark.tex}
\newcommand\ballcircle[1]{%
  {%
    \usebeamercolor{enumerate item}%
    \tikzset{beameritem/.style={circle,inner sep=0,minimum size=2ex,text=enumerate item.bg,fill=enumerate item.fg}}%
    \tikz[baseline=(n.base)]\node(n)[beameritem]{\sffamily#1};%
  }%
}
\newcommand\ballref[1]{%
  \ballcircle{\ref{#1}}%
}

\input{./lib/callout.tex}
\input{./lib/listings.tex}
\newenvironment{notes}
  {%
    \begin{xlrbox}{NotesBox}
    \begin{minipage}{.95\textwidth}
    \small\rmfamily\mcfamily
    \begin{itemize}
    \setlength{\itemindent}{0em}
    \setlength{\footnotesep}{5mm}
  }{%
    \end{itemize}
    \end{minipage}
    \end{xlrbox}
    \note{\theNotesBox}}

\def\AtSOne#1\csod{%
	\begin{array}{c|}
		\hline
		#1\\
		\hline
	\end{array}
}%
\def\AtSTwo#1,#2\csod{%
	\begin{array}{c|c|}
		\hline
		#1 & #2\\
		\hline
	\end{array}
}%
\def\AtSThree#1,#2,#3\csod{%
	\begin{array}{c|c|c|}
		\hline
		#1 & #2 & #3\\
		\hline
	\end{array}
}%
\def\AtSFour#1,#2,#3,#4\csod{%
	\begin{array}{c|c|c|c|}
		\hline
		#1 & #2 & #3 & #4\\
		\hline
	\end{array}
}%
\def\AtSFive#1,#2,#3,#4,#5\csod{%
	\begin{array}{c|c|c|c|c|}
		\hline
		#1 & #2 & #3 & #4 & #5\\
		\hline
	\end{array}
}%
\def\AtSSix#1,#2,#3,#4,#5,#6\csod{%
	\begin{array}{c|c|c|c|c|c|}
		\hline
		#1 & #2 & #3 & #4 & #5 & #6\\
		\hline
	\end{array}
}
\newcommand{\SOne}[1]{\AtSOne#1\csod}
\newcommand{\STwo}[1]{\AtSTwo#1\csod}
\newcommand{\SThree}[1]{\AtSThree#1\csod}
\newcommand{\SFour}[1]{\AtSFour#1\csod}
\newcommand{\SFive}[1]{\AtSFive#1\csod}
\newcommand{\SSix}[1]{\AtSSix#1\csod}
\newcommand\card[2]{%
  \setlength{\fboxsep}{0pt}%
  \fcolorbox{black}{#1}{%
    \hphantom{\rule{0.05em}{0ex}}%
    #2%
    \hphantom{\rule{0.05em}{0ex}}%
    \vphantom{\rule[-0.5ex]{0em}{2.5ex}}%
  }%
}%
\definecolor{coolblack}{rgb}{0.0, 0.18, 0.39}
\newcommand\heartcard{\card{white}{\color{red}{♥}}}
\newcommand\clubcard{\card{white}{\color{coolblack}{♣}}}
\newcommand\commitedcard{\card{gray!20}{\hphantom{\rule{0.28em}{0ex}}?\hphantom{\rule{0.28em}{0ex}}}}%}}
% \def\yescards{\heartcard\,\heartcard\,\clubcard}
% \def\nocards{\heartcard\,\clubcard\,\clubcard}
% \def\threecommitedcards{\commitedcard\,\commitedcard\,\commitedcard}
% \def\threeheartcards{\heartcard\,\heartcard\,\heartcard}
% \def\threeclubcards{\clubcard\,\clubcard\,\clubcard}

\newcommand\ce[1]{%
  \coloremojiucs{#1}
}

\newcommand*{\lstitem}[1]{
  \setbox0\hbox{\lstinline{#1}}
  \item[\usebox0]
}

\presetkeys{todonotes}{inline, noinlinepar}{}

\renewcommand{\arraystretch}{1.2}
\newcolumntype{Y}{>{\centering\arraybackslash}X}

\title[Datatype Generic Programming with Scala 3]{%
  Datatype Generic Programming \\
  with Scala 3
}
\subtitle{}
\author[Yoshimura Hikaru]{%
  \textsc{Yoshimura} Hikaru
}
\Email{hikaru\_yoshimura@r.recruit.co.jp}
\date[April 15-16, 2023]{%
  \oldstylenums{April 15-16, 2023}
}
\Conference[ScalaMatsuri 2023]{Sponsor sessions of ScalaMatsuri 2023}
\institute[\InsertEmail]{Recruit Co., Ltd}
\GitHub{y-yu/scalamatsuri2023}

\newcommand{\facesize}{1cm}
\newcommand\alicecallout[2]{
  \simplecallout[{\includegraphics[width=\facesize]{./img/alice_face.png}}]{#1}{cyan!10}{#2}
}
\newcommand\bobcallout[2]{
  \simplecallout[{\includegraphics[width=\facesize]{./img/bob_face.png}}]{#1}{orange!10}{#2}
}

\begin{document}

\frame{%
  \maketitle
  \note{%
    \vspace*{\fill}
    \centering
    {\LARGE Datatype Generic Programming with Scala 3}
    \vspace{2ex}
    
    \begin{itemize}
      \item こっちのページは日本語の説明で、実際の発表では表示されません

      \item ただオーディエンス向けに事前に配っておく資料的な感じで利用する予定です
    \end{itemize}
    \vspace*{\fill}
  }%
}

\section{Introduction}

\begin{frame}
  \frametitle{Table of contents}

  \tableofcontents
\end{frame}

\begin{frame}
  \frametitle{Who am I?}
  
  \begin{columns}
    \begin{column}{0.3\textwidth}
      \begin{center}
        \begin{figure}
          \includegraphics[width=0.95\textwidth]{img/bird2x.png}
        \end{figure}
      \end{center}
 
      \begin{table}[h]
        \begin{tabular}{ll}
          Twitter & \href{https://twitter.com/\_yyu\_}{@\_yyu\_} \\
          Qiita &  \href{https://qiita.com/yyu}{yyu} \\
          GitHub &  \href{https://github.com/y-yu}{y-yu} \\
        \end{tabular}
      \end{table}
    \end{column}
    \begin{column}{0.7\textwidth}
      \begin{itemize}
        \item Recruit Co., Ltd.
        \begin{itemize}
          \item StudySapuri ENGLISH server side
        \end{itemize}

        \item Quantum Information \& Algorithms

        \item Cryptography \& Security
        
        \item Programming \& {\LaTeX} typesetting
        \begin{itemize}
          \item Scala, Rust, Go, Swift
        \end{itemize}
      \end{itemize}
    \end{column}
  \end{columns}
\end{frame}

\begin{frame}[fragile]
  \frametitle{\lstinline|TestObject|: generating fixtures for unit tests}

  \begin{itemize}
    \item We uses \lstinline|TestObject| on our product,
    which is a utility to generate dummy objects(as known as \emph{fixtures}) for unit tests.
\begin{lstlisting}[style=scala]
case class StudySapuriSession(
  /* very complicated! */
)
val dummyData = TestObject.get[StudySapuriSession]
\end{lstlisting}

    \item Type class \lstinline|TestObject[A]| provides us to the way for generating some value of \lstinline|A|.
    \begin{columns}
      \begin{column}{0.32\textwidth}
\begin{lstlisting}[style=scala]
trait TestObject[A] {
  def generate: State[Int, A]
}
\end{lstlisting}
      \end{column}
      \begin{column}{0.68\textwidth}
\begin{lstlisting}[style=scala]
implicit val strInstance: TestObject[String] = new TestObject {
  def generate: State[Int, A] = State(s => (s + 1, s.toString))
}
\end{lstlisting}
      \end{column}
    \end{columns}

    \item In naive way, we have to define too many \lstinline|TestObject| implicit instances for
    every types used in our product, but it's not possible and reasonable.
  \end{itemize}

  \note{
    \begin{itemize}
      \item このトークではこの\lstinline|TestObject|の話をする

      \item これは単体テストなどのモッキング用に、任意の型\lstinline|A|のダミー値を作成することができる

      \item 実態としてはステートモナドになっており、この\lstinline|Int|のステートを元にして型\lstinline|A|の値を作成する

      \item ナーイブには、このような\lstinline|implicit|インスタンスを、プロダクトに存在する型のぶんだけ
      大量に定義していく必要があるが、そのような作業は無理である
    \end{itemize}
  }
\end{frame}

\section{Overview of datatype generic programming}

\begin{frame}
  \frametitle{\lstinline|TestObject| with datatype generic programming}

  \begin{itemize}
    \item The many \lstinline|TestObject| instances can be provided by
    \emph{datatype generic programming}, not manually.
    \begin{itemize}
      \item We can get \lstinline|dummyData: StudySapuriSession| easily once we define
      \lstinline|TestObject| instances for primitive or Java types,
      \item Then datatype generic programming generates the other instances for our
      defined data structures(= case objects).
    \end{itemize}

    \item Datatype generic programming is the one of the ways of meta-programming.

    \item In this talk I'll explain datatype generic programming with Scala 3.
  \end{itemize}

  \note{
    \begin{itemize}
      \item このようなときに\emph{datatype generic programming}を使うことができる。
      手作業でのインスタンス定義を回避して自動的にインスタンスを提供させる
      \begin{itemize}
        \item ユーザーが手でやるのはプリミティブな型やJavaの型だけでよい
      \end{itemize}

      \item Datatype generic programmingはメタプロの一種である

      \item しかしdatatype generic programmingのやり方がScala 2と3で相当変わってしまった。
      今回のトークではScala 3でのやり方について解説する
    \end{itemize}
  }
\end{frame}

\begin{frame}[fragile]
  \frametitle{Datatype generic programming \textit{vs.} Macros}

  \begin{itemize}
    \item Unfortunately if we would use raw macros, we can create a new syntax like LISP's S-expression in Scala source,
    but we don't want it.
    %\begin{itemize}
    %  \item I think template meta-programming in C++ allows us to prevent creating a new syntax.
    %\end{itemize}

    \item Almost every data structure can be classified either ``tuple'' like or ``enum'' like:
    \begin{columns}
      \begin{column}{0.4\textwidth}
        %, caption=tuple like
\begin{lstlisting}[style=scala]
case class TupleLike(
  field1: Int, field2: String
)
\end{lstlisting}
      \end{column}
      \begin{column}{0.6\textwidth}
        %, caption=enum like
\begin{lstlisting}[style=scala]
sealed trait EnumLike
case class Pattern1(v: Int)    extends EnumLike
case class Pattern2(v: String) extends EnumLike
\end{lstlisting}
      \end{column}
    \end{columns}
    \begin{itemize}
      \item \lstinline|TupleLike| requires both two values of \lstinline|Int| and \lstinline|String|,
      on the other hand \lstinline|EnumLike| requires either \lstinline|Int| value or \lstinline|String| value.
    \end{itemize}

    \item Datatype generic programming provides us following two functions: 
      \ballcircle{1} converting a type value to the analogy tuple or enum and
      \ballcircle{2} reverting it to the original type.
  \end{itemize}

  \note{
    \begin{itemize}
      \item たとえば我々が生のマクロを使うと、Scalaソースコード内でLISPのS式を書けるような構文を創造できるが、
      あまりそれは望まれていない
      \begin{itemize}
        \item 構文が創造されれば、Scalaの他にその創造された構文の勉強も必要になってしまい、
        事実上1つのプログラム言語のために複数の言語を学ぶようなことになる
      \end{itemize}

      \item ほぼ(?)すべてのデータ構造はタプルのような構造と、enumのような構造のどちらかである
      \begin{itemize}
        \item タプルは含まれている型の全ての値が必要で、一方でenumは含まれる型のうちどれかしら1つを要請する
        \item この例だと\lstinline|TupleLike|は\lstinline|Int|と\lstinline|String|の両方が必要だが、
        一方で\lstinline|EnumLike|はどちからがあればインスタンシエイトできる
      \end{itemize}

      \item こういう感じで全てのデータ構造についてタプルとenumに分けていき、
      そのタプルを操作して再び元のデータ構造に戻すみたいな抽象化を提供するのがDatatype generic programmingである
    \end{itemize}
  }
\end{frame}

\begin{frame}[fragile]
  \frametitle{Meta-programming using datatype generic programming}
  
  \begin{itemize}
    \item Almost all meta-programming can be done by such datatype abstraction
    and ad-hoc polymorphism, without probability of creating a new syntax.
    \begin{enumerate}
      \item First convert user defined data structures(= case objects) to tuple or enum like using datatype generic programming.
      \item Then find some implicit instances based on the types included in the tuple or enum.
      \item Finally revert the derived instance for tuple or enum like to one for the original data type.
    \end{enumerate}

    \item \lstinline|case class TupleLike(field1: Int, field2: String)| example:
    \begin{scaledprooftree}{0.8}
      \AxiomC{\texttt{TupleLike $\Leftrightarrow$ (Int, String)}}
      \AxiomC{\lstinline|TestObject[Int]|}
      \AxiomC{\lstinline|TestObject[String]|}
      \RightLabel{{\small \ballcircle{2}}}
      \BinaryInfC{\lstinline|TestObject[(Int, String)]|}
      \RightLabel{{\small \ballcircle{1}}}
      \BinaryInfC{\lstinline|TestObject[TupleLike]|}
    \end{scaledprooftree}
    where {\small \texttt{TupleLike $\Leftrightarrow$ (Int, String)}} is powered by datatype generic programming.
  \end{itemize}

  \note{
    \begin{itemize}
      \item このようにタプルやenumとデータ構造を行きするような抽象化と、アドホック多相があればほぼ全てのメタプログラミングを
      シンタックスの創造なしで行うことができる
      \begin{enumerate}
        \item datatype generic programmingでデータ構造をタプルやenumに変換する
        \item そのタプルやenumに含まれている型のimplicit インスタンスを探索する
        \item そして元々の型に復活させる
      \end{enumerate}

      \item 具体的に\lstinline|TupleLike|で見ていくとこういう感じになる
    \end{itemize}
  }
\end{frame}

\begin{frame}
  \frametitle{Macro compatibility between Scala 2 and 3}

  \begin{itemize}
    \item We use \emph{shapeless}\cite{shapeless_github} for datatype generic programming in Scala 2.

    \item Likewise on our product, we are compiling both of Scala 2 and 3 for almost all code.

    \item There is no compatibility of macros between Scala 2 and 3 \ce{:innocent:}

    \item It follows that \lstinline|TestObject| implemented on Scala 2 won't work well on Scala 3.
    \begin{itemize}
      \item \emph{shapeless 3}\cite{shapeless-3_github} for Scala 3 is being developed but unfortunately
      it doesn't have compatibility of shapeless for Scala 2\ce{:innocent:}
    \end{itemize}

    \item Eventually we(mainly ScalaNinja) began to develop another \lstinline|TestObject| implementation for Scala 3
  \end{itemize}

  \note{
    \begin{itemize}
      \item 我々はScala 2でのdatatype generic programmingにshapelessをつかっている

      \item ところで、我々のプロダクトはScala 3でもほぼ全てのコードをコンパイルしている

      \item そしてScala 2と3でマクロの互換性がない

      \item つまりScala2で作ってあった\lstinline|TestObject|はScala 3では動かない。
      Scala 3対応したshapeless 3も作らてはいるが、これはshapeless 2とは別物というくらいに互換性がない

      \item そういうわけでScala 3版の\lstinline|TestObject|を作ることになった
    \end{itemize}
  }
\end{frame}

\section{Datatype generic programming in Scala 3}

\begin{frame}[fragile]
  \frametitle{Datatype generic programming in Scala 3}

  \begin{itemize}
    \item Scala 3 supports datatype generic programming initially like follows:
    % using
    % \lstinline|summonFrom|, \lstinline|Mirror.ProductOf| and \lstinline|Mirror.SumOf|.
    \begin{columns}
      \begin{column}{0.4\textwidth}
\begin{lstlisting}[style=scala]
import scala.compiletime.*
import scala.deriving.*
case class TupleLike(
  field1: Int, field2: String
)
\end{lstlisting}
      \end{column}
      \begin{column}{0.6\textwidth}
\begin{lstlisting}[style=scala]
scala> Tuple.fromProductTyped(TupleLike(1, "a"))
val res0: (Int, String) = (1,a)

scala> summon[Mirror.ProductOf[TupleLike]].fromProduct(res0)
val res1: TupleLike = TupleLike(1,a)
\end{lstlisting}
      \end{column}
    \end{columns}
    \begin{itemize}
      \item Similarly we can convert \lstinline|sealed trait| to enum like.
    \end{itemize}

    \item We can use some functions to convert case objects from/to tuple like without any libraries.

    \item Meta-programming tools in Scala 3 is reinforced rather than Scala 2\ce{:thumbsup:}
  \end{itemize}

  \note{
    \begin{itemize}
      \item 実はScala 3はライブラリーなしでこのようにケースクラスをタプルに変換するなどの機構が用意されている

      \item こんな感じで\lstinline|scala.compiletime|や\lstinline|scala.deriving|を引っ張ってくることで、
      任意のケースクラスをそのフィールドの型に対応するタプルにしたり、タプルから構成したりできる
      
      \item 今回は紹介しない部分も含めて、Scala 3はメタプログラミングがScala 2に比べて強化されている
    \end{itemize}
  }
\end{frame}

\begin{frame}[fragile]
  \frametitle{\lstinline|TestObject| implementation on Scala 3}

  \begin{itemize}
    \item We'll define \lstinline|derive| method such like:
\begin{lstlisting}[style=scala]
inline implicit def derive[A]: TestObject[A]
\end{lstlisting}   
    \begin{itemize}
      \item \lstinline|derive| provides \lstinline|TestObject| instances for all \lstinline|A|.
    \end{itemize}

    \item This is the overview of \lstinline|derive| behavior:
    \begin{enumerate}
      \item Check if the instance for the input type has been defined.
      \item If not found, pattern match the type into either tuple like or enum like. \label{enum:tuple_or_enum}
      \item Collect the \emph{ill-typed} list of \lstinline|TestObject| for each types contained in \ballref{enum:tuple_or_enum}
      using \lstinline|erasedValue| \label{enum:instances_list}.
      \item Finally make the instance for the input type using \ballref{enum:instances_list} instances list.
    \end{enumerate}

    \item Let's see the details!
  \end{itemize}

  \note{
    \begin{itemize}
      \item 今回の目標はScala 3のメタプロ機構で任意の型\lstinline|A|に対する\lstinline|TestObject[A]|を提供する
      \lstinline|derive|メソッドをつくること

      \item \lstinline|derive|は
      \begin{enumerate}
        \item すでに定義されているインスタンスがないか探して
        \item もしなかったら、タプルかenumのケースへパターンマッチする
        \item \lstinline|erasedValue|をつかって、上記のタプルかenumに入っている型のインスタンスを集めてきて
        型なしリストに詰め込む
        \item そして最後に\ce{:point_up:}のリストで材料をつくって最終的なインプットされた型のインスタンスをつくる
      \end{enumerate}
    \end{itemize}
  }
\end{frame}

\begin{frame}[fragile]
  \frametitle{\ballcircle{1} Check if the instance for the input type has been defined}

  \begin{itemize}
    \item \lstinline|summonFrom| searches the \lstinline|TestObject| instance for type \lstinline|A|
\begin{lstlisting}[style=scala]
inline implicit def derive[A]: TestObject[A] =
  summonFrom {
    case x: TestObject[A] =>
      x
    case _ =>
      create[A] // we'll define next page!
  }
\end{lstlisting}

    \item If \lstinline|summonFrom| finds the \lstinline|TestObject[A]| instance,
    then the instance would be named as \lstinline|x|.
    \begin{itemize}
      \item In this case, it's unnecessary to define the instance so returns \lstinline|x|.
    \end{itemize}

    \item In the latter case, we call \lstinline|create| method to define \lstinline|TestObject[A]|.
  \end{itemize}

  \note{
    \begin{itemize}
      \item \lstinline|summonFrom|をつかってインスタンスを探すことができる

      \item もし見つかったらそれを返せばいい

      \item そうでない場合、\lstinline|create|メソッドをつかって定義していく
    \end{itemize}
  }
\end{frame}

\begin{frame}[fragile]
  \frametitle{\ballcircle{2} Pattern matching if product or sum}

  \begin{itemize}
    \item Since there is no \lstinline|TestInstance[A]| instance yet,
    \lstinline|create| finds \lstinline|ProductOf[A]| or \lstinline|SumOf[A]|
    instance using \lstinline|summonFrom| again.
\begin{lstlisting}[style=scala]
inline final def create[A]: TestObject[A] =
  summonFrom {
    case _: Mirror.ProductOf[A] =>
      deriveProduct[A] // 1
    case _: Mirror.SumOf[A] =>
      deriveSum[A]     // 2
  }
\end{lstlisting}

    \item It means that:
    \begin{enumerate}
      \item \lstinline|A| is a tuple like type (i.e. case classes)
      if there is a \lstinline|ProductOf[A]| instance,
      \item \lstinline|A| is an enum like structure (i.e. sealed traits).
      if there is a \lstinline|SumOf[A]| instance.
    \end{enumerate}
  \end{itemize}

  \note{
    \begin{itemize}
      \item \lstinline|TestInstance[A]|が見つからなかったので、
      \lstinline|create|では再び\lstinline|summonFrom|を使って
      \lstinline|ProductOf[A]|か\lstinline|SumOf[A]|のインスタンスを探す
      \begin{itemize}
        \item もし\lstinline|ProductOf[A]|が見つかったら、型\lstinline|A|は
      ケースクラスなどのタプル的なデータ構造であり、
        \item 一方で\lstinline|SumOf[A]|のインスタンスが見つかれば、
        型\lstinline|A|はsealed traitなどのenum的なデータ構造である
      \end{itemize}
    \end{itemize}
  }
\end{frame}

\begin{frame}[fragile]
  \frametitle{\ballcircle{3} Make \emph{ill-typed} instances list: \lstinline|List[TestObject[?]]|}
  
  \begin{itemize}
    \item Before see \lstinline|deriveProduct| and \lstinline|deriveSum|,
    we have to prepare the way to collect all instances for types being contained in \lstinline|A|.
    \begin{columns}
      \begin{column}{0.55\textwidth}
        \begin{itemize}
          \item For example \lstinline|TupleLike|, 
          we need the both instances of \lstinline|TestObject[Int]| and \lstinline|TestObject[String]|.
        \end{itemize}
      \end{column}
      \begin{column}{0.35\textwidth}
\begin{lstlisting}[style=scala]
case class TupleLike(
  field1: Int, field2: String
)
\end{lstlisting}
      \end{column}
    \end{columns}

    \item \lstinline|erasedValue| allows us to search and collect all instances recursively.
    \begin{columns}
      \begin{column}{0.6\textwidth}
\begin{lstlisting}[style=scala]
inline def deriveRec[T <: Tuple]: List[TestObject[?]] =
  inline erasedValue[T] match {
    case _: EmptyTuple =>
      Nil
    case _: (t *: ts) =>
      derive[t]/* mutual recursion */ :: deriveRec[ts]
  }
\end{lstlisting}
      \end{column}
      \begin{column}{0.3\textwidth}
        \item There is no type compatibility among the instances,
        \lstinline|deriveRec| cannot help but to return \emph{ill-typed} list\ce{:innocent:}
      \end{column}
    \end{columns}

    \item In addition \lstinline|*:| is type-level tuple constructor
    provided since Scala 3.
  \end{itemize}
\end{frame}

\begin{frame}[fragile]
  \frametitle{\ballcircle{4a} \lstinline|deriveProduct| case}
  
  \begin{itemize}
    \item Using \lstinline|deriveRec|, we define \lstinline|TestObject| instance for \lstinline|A|
    in \lstinline|deriveProduct| case.
\begin{lstlisting}[style=scala]
inline def deriveProduct[A](using a: ProductOf[A]): TestObject[A] = {
  def p: TestObject[A] = {
    val xs = deriveRec[a.MirroredElemTypes]
    productImpl[A](xs, a)
  }
  p
}
\end{lstlisting}
    \begin{itemize}
      \item Why does \lstinline|deriveProduct| only call \lstinline|productImpl|
      through temporary method \lstinline|p|?\ce{:thinking:}
    \end{itemize}

    \item This is the ScalaNinja's remarkable and state-of-the-art technique to
    avoid
    \begin{itemize}
      \item throwing \lstinline|MethodTooLargeException| due to \lstinline|inline|
      \item and generating too many nameless classes instead of nameless class.
    \end{itemize}

    \item In meta-programming, we have to consider about compiling efficiency, not only runtime.
    That's maybe the why meta-programming is difficult\ce{:innocent:}
  \end{itemize}
\end{frame}

\begin{frame}[fragile]
  \frametitle{\ballcircle{4a} \lstinline|deriveProduct| case}
\end{frame}

\begin{frame}[fragile]
  \frametitle{\ballcircle{3b} \lstinline|Sum| case}

  \begin{itemize}
    \item In \lstinline|Sum| case, we generate a value in \lstinline|values|.
\begin{lstlisting}[style=scala],
  final def sumImpl[A](values: List[DeterministicTestObject[?]]): DeterministicTestObject[A] =
    new DeterministicTestObject[A] {
      def generate: IntState[A] =
        for {
          s <- State.get
          value = values(values.length % s)
          result <- value.generate
        } yield result
    }
\end{lstlisting}
  \end{itemize}

  \note{
    \begin{itemize}
      \item TODO: もうちょっと英語がんばる
    \end{itemize}
  }
\end{frame}

\section{Conclusion}

\begin{frame}
  \frametitle{Conclusion}

  \pause
  \begin{itemize}
    \item<+-> There is no macro compatibility between Scala 3 and Scala 2 \ce{:innocent:}
    \begin{itemize}
      \item And shapeless 2 and 3 don't have the same interface.
    \end{itemize}

    \item Scala 3 supports datatype generic programming initially.

    \item Happy meta-programming!
  \end{itemize}
\end{frame}

\section*{References}
\begin{frame}%[allowframebreaks]
  \frametitle{References}
  % \nocite{*}
  \bibliographystyle{junsrt_url}
  \bibliography{ref}
\end{frame}

\begin{frame}
  \centering
  {\Huge Thank you for the attention!}
\end{frame}

\end{document}
